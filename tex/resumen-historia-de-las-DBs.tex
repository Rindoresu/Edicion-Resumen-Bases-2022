\section*{Historia de las Bases de Datos (posiblemente no evaluado 2022)}

Este tema no parece ser evaluado en los finales (no lo vi preguntado
en los de entre 2018-2022), pero quizá sirve.

\begin{itemize}
    \item Origen 1960, Network Data Model (Integrated Data Store y CODASYL)
    \item En la  misma época, Modelos Jerárquicos (IBM Information Management System)
    \item Ted Codd 1970s, Modelo Relacional
    \item 1980s Boom Relacional (System R, Ingres, Oracle) y Relational-Object Mismatch Problem.
    \item 1990s PostgreSQL, MySQL, Microsoft SQL Server (Para usuarios
    más pequeños de Bases de Datos).
    \item 2000s NoSQL boom por el crecimiento de la web y su demanda, middleware (Facebook Google), Datawarehousing y OLAP (a diferencia de OLTP, ambos vistos en su sección respectiva).
    \item 2010s NewSQL Boom de 2 tipos: Por un lado DBMS OLTP ACID distribuidos, por otro lado
Hibridos OLTP y OLAP distribuidos mezcla open y closed source.
    \item 2010s Tambien aparece las DBaas (DataBase as a Service), Bases de Grafos y de datos temporales.
\end{itemize}