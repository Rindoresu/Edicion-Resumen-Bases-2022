\section*{Datos semi-estructurados y XML}

Los datos estructurados no fueron muy compatibles con el alza de la web, \textbf{HTML}, y otros.

Los datos estructurados tienen un formato estricto, fijo y predefinido pueden relacionarse con otros datos en registros de la misma estructura.

Hacer traducciones constantes de \textbf{HTML} a lenguajes estructurados tiene el problema de volver \textbf{frágil} y propenso a que \textbf{deje de funcionar} los programas, y requerir HTTP requests incluso cuando solo se requiere un valor de una tabla.

Los \textbf{datos semi-estructurados} tienen cierta estructura pero \textbf{no garantizan la predictibilidad y organización} de los \textbf{estructurados}. Aún así, \textbf{los semi-estructurados}, tienen marcas y otros elementos que les permite jerarquizarlos, pero por tanto \textbf{la definición de la estructura esta mezclada con los datos}.

\texttt{XML, eXtensible Markup Language} es un lenguaje parecido a \textbf{HTML} pero no igual, que permite intercambiar datos con cierta estructura \textbf{(semi-estructurado)} como \textbf{JSON} o los \textbf{csv}.
\textbf{Permite representar datos más amigables} para el formato Humano-Computadora, con \textbf{la desventaja} que \textbf{hay mucha redundancia} y \textbf{repetición de los datos a diferencia de los estructurados}.

\textbf{DTD} (Document Type Definition) y \textbf{XML Schema} permiten definir estructuras en \textbf{XML} con el objetivo de asegurar que los archivos \textbf{XML} recibidos o enviados conforman un cierto esquema, \textbf{permiten además chequeos automáticos que verifican que el archivo cumple con dicho esquema}.

\textbf{XPath} es un lenguaje de recuperación y consulta de archivos \textbf{XML}, utiliza jerarquización en base a directorios como una estructura de carpetas.

\textbf{XQuery} es un lenguaje puramente de consultas, que toma conceptos de \textbf{SQL}, \textbf{XQL} y otros lenguajes de consulta para \textbf{XML}.

Además, existen extensiones para \textbf{SQL} que permiten tratar con archivos \textbf{XML} como por ejemplo la extensión de \textbf{Oracle} (y así poder usar \textbf{datos semi-estructurados} en \textbf{entornos estructurados}).