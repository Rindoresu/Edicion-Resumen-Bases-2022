\section*{Lenguajes de consulta}
\subsection*{Álgebra relacional (AR)}
Es un lenguaje formal especificado en base a propiedades matemáticas que en el modelo relacional le permite a los usuarios especificar consultas sobre instancias de relaciones, dando como resultado una nueva relación. Permite así formalizar las operaciones asociadas al modelo relacional, da una base para implementar y optimizar consultas en RDBMS y constituye los módulos internos de las principales operaciones y funciones de muchos sistemas relacionales. \\
Es \textbf{procedural y axiomático} con operaciones unarias y binarias.

\subsubsection*{Operaciones unarias}
\begin{itemize}
    \item \textbf{Select($\sigma$):} Selecciona un subconjunto de tuplas de una relación en base a cierta condición. Genera una \textbf{partición horizontal}, es conmutativa, se puede agrupar en cascada y conserva la cantidad de atributos. En SQL:
    \begin{SQL}
        SELECT $*$ FROM $Relacion$ WHERE $Condicion$
    \end{SQL}
    \item \textbf{Project($\pi$):} Selecciona un subconjunto de columnas de una relación. Genera una \textbf{partición vertical}, no es conmutativa y la cantidad de tuplas se preserva si la proyección contiene alguna superclave. En SQL:
    \begin{SQL}
        SELECT DISTINCT $Columnas$ FROM $Relacion$
    \end{SQL}
    \item \textbf{Rename($\rho$):} Le asigna un nuevo nombre a los atributos y/o a la relación. Suele usarse para resultados intermedios. En SQL:
    \begin{SQL}
        SELECT $NRelacion.Columna$ AS $NColumna$ FROM $Relacion$ AS $NRelacion$
    \end{SQL}
\end{itemize}

\subsubsection*{Operaciones binarias (entre R y S)}
\begin{itemize}
    \item \textbf{Union/Intersection/Minus:} Se corresponden respectivamente con las operaciones matemáticas de conjuntos $\cap$, $\cup$ y $\backslash$. Su relación resultante no contiene duplicados y requiere que R y S sean \textbf{union compatibles}, es decir, que contengan la misma cantidad de atributos y sus tipos de datos coincidan en orden. La relación resultante contendrá los nombres de R. En SQL (se puede agregar \texttt{ALL} para no eliminar duplicados):
    \begin{SQL}
        SELECT $Columnas$ FROM $R$ UNION/INTERSECT/MINUS SELECT $Columnas$ FROM $S$
    \end{SQL}
    \item \textbf{Producto Cartesiano:} Crea una nueva relación que combina cada tupla de R con una de S. Su grado es la suma de los grados de sus relaciones y no requiere que las relaciones sean union compatibles. En SQL:
    \begin{SQL}
        SELECT $*$ FROM $R$ CROSS JOIN $S$
    \end{SQL}
    \item \textbf{Join:} Combina pares de tuplas relacionadas entre R y S en base a cierta condición. Esto no incluye a las tuplas NULL ni a las que no cumplan con la condición (que debe tener cierto formato para ser válida). Es conmutativa, su grado es la suma de los de sus relaciones y la cantidad de tuplas resultantes depende de la condición. Si la condición es de igualdad se denota \textbf{EquiJoin}. En SQL se usa con las cláusulas \texttt{SELECT}, \texttt{FROM} y \texttt{WHERE}.
    \item \textbf{Natural Join:} Similar a Join pero relacionando los campos del mismo nombre, dejando sólo uno de los duplicados. Para ello requiere de la correspondencia de nombres o que se haga un renombre previo.
    \item \textbf{Outer Join:} A diferencia de los anteriores clasificados como \textbf{Inner Join}, aquí se pueden incorporar al resultado las tuplas de R, S o ambas que no cumplan con la condición. Respectivamente son de tipo \textbf{Left}, \textbf{Right} y \textbf{Full}, y los atributos restantes de la tupla en el resultado se rellenan con NULL. Estas operaciones son parte del estándar SQL2.
    \item \textbf{Division:} Retorna los valores de R emparejados con todos los valores de S, para lo cual requiere que los de S estén en los de R. El resultado contiene los atributos en R que no están en S. Si bien no suele implementarse en SQL puede expresarse en base a otras operaciones.
\end{itemize}

\subsection*{CRT (Cálculo Relacional de Tuplas)}
Es otro lenguaje de consultas asociado al MR. Emplea una \textbf{técnica declarativa} de consultas (no describe un orden de evaluación) con fundamentos basados en la lógica matemática y forma las bases fundacionales de SQL. Se basa en la expresión:
\begin{SQL}
    \{t | COND(t)\}
\end{SQL}
donde \texttt{t} es una tupla y la única variable de la expresión, y \texttt{COND} es una fórmula bien formada de CRT. Como resultado se obtiene un conjunto de todas las tuplas de CRT que verifican la condición.
