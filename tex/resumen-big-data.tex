\section*{Big Data}

Big Data se los define a los datos que cumplen con las \textbf{5 V's} en mayor o menor medida.

\begin{itemize}
    \item \textbf{Volumen:} presentan un gran \textbf{volúmen} de datos.
    \item \textbf{Velocidad:} accesibles a \textbf{velocidades} que permiten el acceso en cuestión de milisegundos (cómo es el caso de los mercados bursátiles).
    \item \textbf{Variedad:} con datos de gran \textbf{variedad}, de muchos entornos distintos y diversos, que pueden estar estructurados o no, venir de paradigmas varios, e integrar muchos modelos.
    \item \textbf{Veracidad:} donde los datos pueden ser poco confiables, ruidosos, mezclar orígenes de datos que al combinarlos, su output puede generar incertidumbre, y que por tanto es necesario tratarlos, analizarlos, categorizarlos según fiabilidad, con el fin de poder corroborar con algún criterio su \textbf{Veracidad}.
    \item \textbf{Valor:} teniendo en cuenta que los datos proveen valor intrínseco al permitir tomar decisiones eficientes y precisas que le otorgan \textbf{Valor}.
\end{itemize}