\section*{Modelización}
Los pasos para diseñar una base de datos relacional son:
\begin{center}
    Requerimientos $\leftrightarrow$ MER $\leftrightarrow$ MR $\leftrightarrow$ Normalización $\leftrightarrow$ Diseño físico $\leftrightarrow$ BD
\end{center}
Se basan en la metodología de diseño lógico para bases de datos (LRDM). \\
\textbf{MC (Modelo Conceptual):} Es una concpetualización formal del mundo real (dominio específico) que modela sus objetos, características y relaciones. Se usan para comunicar ideas, buscar concensos y tienen gran importancia a la hora de analizar aplicaciones y validar acciones de los usuarios. Se representan a través de lenguajes de ontologías, generalmente expresados a través de diagramas (DERs, UML).

\subsection*{MER (Modelo Entidad-Relación)}
Es una herramienta que permite realizar una abstracción o modelo de alguna situación de interés del mundo real. Se realiza a través de la técnica de los \textbf{DERs} (Diagramas de Entidad-Relación) conformados por los siguientes elementos.

\subsubsection*{Entidades}
Objeto o concepto del que queremos registrar información en un contexto dado. Se representan a través de bloques y conforman conjuntos de instancias. Pueden ser:
\begin{itemize}
    \item \textbf{Fuertes:} Su existencia no depende de la de otra entidad y se identifican a través de atributos propios.
    \item \textbf{Débiles:} Necesitan de la identificación de otra entidad para distinguirse.
\end{itemize}
Además cuando tenemos varias que comparten atributos podemos generalizarlas de forma parcial o total, tratando a cada una como una \textbf{especialización}. Estas heredan atributos de la generalización y pueden ser disjuntas (con atributo discriminante) o solapadas.

\subsubsection*{Atributos}
Son las características descriptivas de las entidades relevantes al problema. Con ello constituyen la información concreta a mantener de cada elemento de una entidad. Tienen un respectivo dominio y pueden identificar a la instancia. Según cómo lo hagan tenemos:
\begin{itemize}
    \item \textbf{Superclaves (SK):} Conjunto de atributos que identifican unívocamente las instancias.
    \item \textbf{Claves candidatas (CK):} SK minimales.
    \item \textbf{Clave primaria (PK):} Una CK elegida según un criterio para identificar a la entidad.
\end{itemize}
Además pueden ser \textbf{multivaluados} (más de un valor a la vez) y \textbf{compuestos} (se dividen en subpartes con significados diferentes), los cuales se usan poco en la realidad.

\subsubsection*{Interrelaciones}
Constituyen la manera de vincular entidades en el dominio del problema. Su \textbf{cardinalidad} determina la cantidad de instancias participantes de cada entidad (1-1, 1-N, N-M) y su \textbf{grado} la cantidad de entidades (unarias, binarias y ternarias). Cada extremo de la relación se corresponde con un \textbf{rol}. Además la \textbf{participación} de las entidades puede ser total o parcial según si todas sus instancias deben formar parte de ella. Se almacenan como n-uplas ordenadas. \\
Cuando se desea interrelacionar una relación N-M con una entidad se puede usar una \textbf{agregación} que abstraiga a la primera como una entidad.

\subsubsection*{Reglas de dominio}
Son restricciones adicionales destinadas a expresar limitaciones sobre los datos que no pueden expresarse de otra manera. Pueden escribirse a través de lenguajes formales o informales con tal de salvar la pérdida de información.

\subsubsection*{Trampas de conexión}
En caso de haber una mala interpretación de las interrelaciones que genere que se pierda información y no se tenga una representación adecuada del mundo real decimos que tenemos una \textbf{trampa de conexión}. Entre ellas tenemos:
\begin{itemize}
    \item \textbf{Trampa del abanico (Fan Traps):} El camino entre ciertas entidades es ambiguo. Sucede generalmente cuando salen dos o más interrealciones 1:N en abanico desde la misma entidad.
    \item \textbf{Trampa del sumidero (Chasm Traps):} La interrelación existente entre dos entidades del modelo no tiene camino. Suele aparecer cuando una entidad participa parcialmente de dos o más relaciones.
\end{itemize}
Para solucionarlas suele reestructurarse el modelo.

\subsubsection*{Otras consideraciones sobre DERs}
\begin{itemize}
    \item La cardinalidad de las interrelaciones ternarias se basa en ver la cantidad de elementos de una entidad que se relacionan con un par de las otras.
    \item Las interrelaciones unarias deben tener al menos un extremo de participación parcial y sus roles definidos.
    \item No suelen modelarse entidades de un sólo registro.
    \item Las interrelaciones N-M pueden tener atributos identificatorios para tener mas de un vínculo entre instancias con distinto atributo (registro, historial).
    \item Según el contexto pueden reemplazarse interrelaciones ternarias por pares de binarias o agregaciones.
    \item Para evitar proliferación de roles en las especializaciones se puede tener una entidad aparte.
\end{itemize}

\subsection*{MR (Modelo Relacional)}
Es la herramienta que nos permite expresar el \textbf{esquema lógico} de la base de datos. Se basa en un conjunto de \textbf{relaciones} que pueden pensarse como \textbf{tablas} con columnas y filas. Las columnas se corresponden con atributos y su conjunto conforma el esquema de la tabla, tomando el dominio de cada uno. Por otra parte las filas conforman las instancias en base a un conjunto de valores (registros). \\
Cada entidad se verá representada por una relación y para representar las interrelaciones usamos el concepto de \textbf{clave foránea (FK)} correspondiente a la PK de de otra entidad. Sobre esto, tenemos las \textbf{restricciones adicionales de referencia e integridad} para que las relaciones representen de manera consistente al esquema lógico.
