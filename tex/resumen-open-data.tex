\section*{OpenData}
Es un \textbf{movimiento} que apoya la \textbf{difusión de los datos} por parte de los Estados, organismos internacionales y empresas para promover su análisis y utilización en \textbf{formatos abiertos} para crear visualizaciones, aplicaciones y herramientas. \\
Los datos son publicados con dos motivos principales:
\begin{itemize}
    \item Transparencia y participación ciudadana (open goverment, repercutió en América Latina)
    \item Generación de servicios para empresas y de valor para iniciativas privadas (repercutió en Europa)
\end{itemize}
Las publicaciones son públicas o semi-públicas y son accedidas por parte del periodismo, ongs, emprendedores y entidades académicas o estudiantiles. El acceso se considera \textbf{abierto} si al estar a disposición del público los datos cumplen con los siguientes principios:
\begin{enumerate}[label=\roman*]
    \item \textbf{Completo:} Todos están disponibles, sin limitaciones de seguridad o privilegio.
    \item \textbf{De primera fuente:} Son coleccionados desde la fuente, con el mayor grado posible de granularidad, sin ningún tipo de modificación o agregación.
    \item \textbf{En tiempo:} Están disponibles tan pronto como sea necesario para preservar su valor.
    \item \textbf{Accesible:} Están disponibles para el rango más amplio de usuarios y propósitos.
    \item \textbf{Procesables por computadoras:} Están estructurados razonablemente como para permitir su procesamiento automático.
    \item \textbf{No discriminatorios:} Están disponibles para todos, sin necesidad de registro (pueden ser accedidos de forma anónima, incluso a través de proxies anónimos).
    \item \textbf{No propietario:} Tienen que tener un formato del que ninguna entidad tenga control exclusivo.
    \item \textbf{Sin licencia:} No deben estar sujetos a derechos de autor, patente, etc.
\end{enumerate}
